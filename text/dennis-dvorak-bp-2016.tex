%%%%%%%%%%%%%%%%%%%%%%%%%%%%%%%%%%%%%%%%%%%%%%%%%%%%%%%%%%%%%%%%%%%%%%%%%%%%%%%%%%%%%%%%%%%%%%%%%%%%%%%
%%													%%
%% 	BAKALÁŘSKÁ PRÁCE - dopnit                                         				%%
%% 				 Dennis Dvořák   							%%
%%													%%
%% pro formátování využita šablona: http://geo3.fsv.cvut.cz/kurzy/mod/resource/view.php?id=775 	%%
%%													%%
%%%%%%%%%%%%%%%%%%%%%%%%%%%%%%%%%%%%%%%%%%%%%%%%%%%%%%%%%%%%%%%%%%%%%%%%%%%%%%%%%%%%%%%%%%%%%%%%%%%%%%% 

\documentclass[%
  12pt,         			% Velikost základního písma je 12 bodů
  a4paper,      			% Formát papíru je A4
  twoside,       			% Oboustranný tisk
  pdftex				    % překlad bude proveden programem 'pdftex' do PDF
]{report}       			% Dokument třídy 'zpráva'
%

\usepackage[czech, english]{babel}	% použití češtiny, angličtiny
\usepackage[utf8x]{inputenc}		% Kódování zdrojových souborů je UTF8

\usepackage[square,sort,comma,numbers]{natbib}

\usepackage{caption}
\usepackage{subcaption}
\captionsetup{font=small}
\usepackage{enumitem} 
\setlist{leftmargin=*} % bez odsazení

\makeatletter
\setlength{\@fptop}{0pt}
\setlength{\@fpbot}{0pt plus 1fil}
\makeatletter

\usepackage[dvips]{graphicx}   
\usepackage{color}
\usepackage{transparent}
\usepackage{wrapfig}
\usepackage{float} 

\usepackage{cmap}           
\usepackage[T1]{fontenc}    

\usepackage{textcomp}
\usepackage[compact]{titlesec}
\usepackage{amsmath}
\addtolength{\jot}{1em} 

\usepackage{chngcntr}
\counterwithout{footnote}{chapter}

\usepackage{acronym}
\usepackage{listings}
\lstset{language=XML,
  literate={ý}{{\'y}}1,
  stringstyle=\color{black},
  identifierstyle=\color{darkblue},
  keywordstyle=\color{cyan},
  morekeywords={xmlns,version,type}}

\usepackage[
    unicode,                
    breaklinks=true,        
    hypertexnames=false,
    colorlinks=true, % true for print version
    citecolor=black,
    filecolor=black,
    linkcolor=black,
    urlcolor=black
]{hyperref}         

\usepackage{url}
\usepackage{fancyhdr}

\usepackage[
  cvutstyle,          
  bachelor           
]{thesiscvut}


\newif\ifweb
\ifx\ifHtml\undefined % Mimo HTML.
    \webfalse
\else % V HTML.
    \webtrue
\fi 

\renewcommand{\figurename}{Obrázek}
\def\figurename{Obrázek}

%%%%%%%%%%%%%%%%%%%%%%%%%%%%%%%%%%%%%%%%%%%%%%%%%%%%%%%%%%%%%%%%%
%%%%%%%%%%% Definice informací o dokumentu  %%%%%%%%%%%%%%%%%%%%%
%%%%%%%%%%%%%%%%%%%%%%%%%%%%%%%%%%%%%%%%%%%%%%%%%%%%%%%%%%%%%%%%%

%% Název práce
\nazev{Nástroj pro práci s daty RÚIAN v programu QGIS}{ QGIS plugin for RUIAN data processing}

%% Jméno a příjmení autora
\autor{Dennis}{Dvořák}

%% Jméno a příjmení vedoucího práce včetně titulů
\garant{Ing. Martin Landa, PhD.}

%% Označení oboru studia
\oborstudia{Geodézie, kartografie a geoinformatika}{}

%% Označení ústavu
\ustav{Katedra geomatiky  }{}

%% Rok obhajoby
\rok{2016}

%Mesic obhajoby
\mesic{červen}

%% Místo obhajoby
\misto{Praha}

%% Abstrakt
\abstrakt 
{Cílem bakalářké práce je návrh nástroje (tzv. zásuvného modulu) do open source projektu  QGIS umožňujícího uživatelsky přívětivou práci s daty RÚIAN (Registr územní identifikace, adres a nemovitostí) poskytovanými ve výměnném formátu VFR v rámci veřejného dálkového přístupu provozovaného ČÚZK. Vzniknuvší softwarový nástroj bude spojovat do funkčního celku dva doposud oddělené projekty laboratoře OSGeoREL na ČVUT v Praze a to návrh grafického uživatelského rozhraní zásuvného modulu a sady konzolových nástrojů pro dávkové stahování a import dat VFR do prostředí GIS.}%
{english/ abstract}

%% Klíčová slova
\klicovaslova
{OSM, import, RUIAN, python, GDAL, GIS, QGIS, pluggin}%
{OSM, import, RUIAN, python, GDAL, GIS, QGIS, pluggin}

%%%%%%%%%%%%%%%%%%%%%%%%%%%%%%%%%%%%%%%%%%%%%%%%%%%%%%%%%%%%%%%%%%%%%%%%

%%%%%%%%%%%%%%%%%%%%%%%%%%%%%%%%%%%%%%%%%%%%%%%%%%%%%%%%%%%%%%%%%%%%%%%%
%% Nastavení polí ve Vlastnostech dokumentu PDF
%%%%%%%%%%%%%%%%%%%%%%%%%%%%%%%%%%%%%%%%%%%%%%%%%%%%%%%%%%%%%%%%%%%%%%%%
\nastavenipdf
%%%%%%%%%%%%%%%%%%%%%%%%%%%%%%%%%%%%%%%%%%%%%%%%%%%%%%%%%%%%%%%%%%%%%%%

%%% Začátek dokumentu
\begin{document}

\catcode`\-=12  % pro vypnuti aktivniho znaku '-' pouzivaneho napr. v \cline 

% aktivace záhlaví
\zahlavi

% předefinování vzhledu záhlaví
\renewcommand{\chaptermark}[1]{%
	\markboth{\MakeUppercase
	{%
	\thechapter.%
	\ #1}}{}}

% Vysázení přebalu práce
%\vytvorobalku

% Vysázení titulní stránky práce
\vytvortitulku

% Vysázení listu zadani
\stranka{}%
	{\sffamily\Huge\centering\ ZDE VLOŽIT LIST ZADÁNÍ}%
	{\sffamily\centering Z~důvodu správného číslování stránek}

% Vysázení stránky s abstraktem
\vytvorabstrakt

% Vysázení prohlaseni o samostatnosti
\vytvorprohlaseni

% Vysázení poděkování
\stranka{%nahore
       }{%uprostred
       }{%dole
       \sffamily
	\begin{flushleft}
		\large
		\MakeUppercase{Poděkování}
	\end{flushleft}
	\vspace{1em}
		%\noindent
	\par\hspace{2ex}
	{Jako první bych chtěl poděkovat vedoucímu mé bakalářské práce, Ing. Martinu Landovi, PhD., za odborné rady, konzultace a především trpělivost během zpracování této práce. Dále bych rád poděkoval kolegům z geodetické kanceláře Nedoma & Řezník a především Ing. Petru Jaškovi a RNDr. Janu Barešovi za připomínky k této bakalářské práce a v neposlední řadě bych chtěl poděkovat své rodině a přítelkyni za projevenou podporu a trpělivost.}

% Vysázení obsahu
\obsah

% Vysázení seznamu obrázků
\seznamobrazku

% Vysázení seznamu tabulek
%\seznamtabulek

% jednotlivé kapitoly
\chapter{Úvod}
\label{1-uvod}

Cílem této práce je...

...bude ještě doplněn...


\chapter{RÚIAN (Registr územní identifikace, adres a nemovistostí)}
\label{2-ruian}

RÚIAN neboli registr územní identifikace, adres a nemovistostí....

...bude ještě doplněn...


\include{3-import}

%\include{9-zaver}

% Vysázení seznamu zkratek
%\include{zkratky}

% Literatura
\nocite{*}
\def\refname{Literatura}
\bibliographystyle{mystyle}
%\bibliography{literatura}


% Začátek příloh
%\prilohy

% Vysázení seznamu příloh
%\seznampriloh

% Vložení souboru s přílohami
%\include{prilohy}

% Konec dokumentu
\end{document}
